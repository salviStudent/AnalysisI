

\documentclass{article}
\usepackage[utf8]{inputenc}
\usepackage{amsmath}
\usepackage{amsfonts}
\usepackage{amssymb}
\usepackage{titlesec}
\usepackage[left=4cm,right=4cm,top=4cm,bottom=4cm]{geometry}

\usepackage{fancyhdr}
\titlelabel{\thetitle\enspace}

\begin{document}
\title{Analysis I Chapter 5 Exercises}
\author{Vladimir Guevara-Gonzalez}
\maketitle
\thispagestyle{fancy}
\begin{enumerate}
\item[5.1] Cauchy Sequences
  \begin{enumerate}
  \item[5.1.1] Let \(a_{n}\) be a cauchy sequence. By definintion
    \(a_{n}\) is 1-steady, hence \(\exists\)N such that
    \(\forall i,j \geq\)N \(\left|a_{i} - a_{j}\right| \leq 1\). If
    we fix \(j\) then from the reverse triangle inequality we
    obtain \(\left| a_{i}\right| - \left|a_{j}\right| \leq 1\) which
    implies \(\left|a_{i}\right| \leq 1 + \left|a_{j}\right|\),
    which shows \(a_{i}\) is bounded by
    \(1+\left|a_{j} \right|\)  \(\forall i \geq N\).Consider all the
    terms before the index \(N\) of \(a_{n}\), these make up a
    up a finite sequence which is bounded by \(x =\) max(\(\left \{a_{k} | k < N\right \}\)). Let \(M = \)max(\(x, 1+a_{j}\)) then
    \(a_{n} \leq M \) \(\forall \)n.
  \end{enumerate}
\item[5.2] Equivalent Cauchy Sequences
  \begin{enumerate}
    \item[5.2.1]
  \end{enumerate}
\item[5.3] The Construction of The Real Numbers
  \begin{enumerate}
  \item[5.3.1]
  \end{enumerate}
\item[5.4] Ordering The Real Numbers
  \begin{enumerate}
  \item[5.4.1]
  \end{enumerate}
  \item[5.5] The Least Upper Bound Property
  \begin{enumerate}
    \item[5.5.1]
  \end{enumerate}
  \item[5.6] Real Exponentiation Part I
  \begin{enumerate}
    \item[5.6.1]
  \end{enumerate}
\end{enumerate}



\end{document}
